\subsection{Interview mit Dr. Hüttenhofer}
\begin{center}
\begin{tabular}{p{6cm}|p{9.5cm}}
Fragen & Antworten\\
\hline
Sie haben im Gebiet der Kunststoffe (Katalysatoren) studiert, wie sind Sie zum Entschluss gekommen sich politisch zu äußern? &Er sei schon immer politisch interessiert gewesen, jedoch
ist er erst seit 3 Jahren Mitglied einer Partei (die Grünen)
Partei ergriffen, hat er weil er sich immer stärker dazu
gedrängt fühlte mit seinem Fachwissen weiter zu helfen.\\
\hline
Würden Sie mit dem jetzigen Wissen darüber, was Kunststoffe anrichten können trotzdem wieder in diesem Bereich studieren?  & \glqq Ja, jedoch bestimmte Forschung nicht im Auftrag von ethnischen "no go's". Polymeer werkstoffe werden wir immer brauchen. Durch die Kreislaufwirtschaft kann man sie wieder abbauen, sprich man sollte sie recyclen. Das Problem ist nicht die Forschung, sondern der Umgang, weshalb man mehr chemische Prozesse für umweltfreundlichere Stoffe machen sollte.\grqq{}\\
\hline
Was sagen Sie zur Situation in Australien (Waldbrände)? & \glqq Der Ursprung von einem Waldbrand ist multikausal die Intensität wird jedoch klar durch den Klimawandel verstärkt.\grqq{}\\
\hline
BPA war Urprünglich als ein hormoneller Stoff mit einer Östrogenen Wirkung vorgesehen, was halten Sie davon bzw. was hätte man besser machen können?&Er ist Überrascht, es zeigt eine Lücke in der Auswertung. Gegenwärtig gibt es eine liste der "Reach" für gefährliche Stoffe.\\
\hline
Dieses Jahr wurde BPA in Thermopapieren für verboten erklärt, reicht dieses Verbot oder wie weit sollten Verbote ihrer Meinung nach gehen? & \glqq Die Wirkung auf die Umwelt ist unbedenklich, weshalb man einen bestimmten Grenzwert haben sollte. Hierbei, sollte wissenschaftlich ran gegangen werden, indem man sich die Anwendungsgebiete ansieht und die Risiken von dem Nutzen abwägt. Außerdem, sollte es strenge Verbote geben um so viel wie möglich zu verbieten. Zum Beispiel sollte es im Verpackungsbereich verboten werden.\grqq{}\\
\hline
Wie nachhaltig, sehen Sie den Umgang mit Kunststoffen in Deutschland? Und was sollte schnellstmöglich verändert werden? & \glqq Die Menge an hergestellten Produkten ist eindeutig zu
extrem und im Verbrauch sind wir deutschen schlecht,
da die Wiederverwendung meist nicht wirtschaftlich ist.
Man sollte hergestelle Produkte solange wie möglich
verwenden, recycling verbessern und monomere
wiederherstellen $->$ Kreislaufwirtschaft, außerdem
sollten abgaben auf Kunststoffmüll und die CO2 Steuer
ausgedehnt werden.\grqq{}\\
\hline
Was belastet Sie in ihrer Arbeit als (Ehrenamtlicher) Politiker am meisten? & \glqq Schon alleine das notwendigste zu erreichen stellt
eine Herausforderung, da Interessen oft emotional oder irrational sind.
"Fake" news sind ein Problem.
Es braucht sehr radikale Maßnahmen, jedoch bekommt
dadurch weniger Wähler. Man soll das Ziel vor den
Augen nicht verlieren. Die Arbeit als Politiker ist
oft frustrierend.\grqq{}
\end{tabular}
\end{center}
