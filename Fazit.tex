\section{Fazit}
\subsection{Zusammenfassung}
Angefangen mit den Biochemikern Edward C. Dodds und Wilfrid Lawson die im Jahr 1936 einen Ersatzstoff für Östrogen suchten ist Bisphenol A mitlerweile nicht mehr aus dem Alltag herauszudenken. Der großteil des hergestellten BPA wird zwar zu stabilen Kunststoffen verarbeitet,jedoch lässt sich unter gewissen Bedingungen die Chemikalie lösen, zum Beispiel bei zu starker Erhitzung in Verbindung mit Wasser und/oder Spülmittel. Auch mechanisch durch durch die maschinelle Reinigung von Polycarbonat-Flaschen, ist eine Freisetzung aus zu schließen.\\
\\
Durch die Freisetzung in den Menschlichen Körper kann es erhebliche Schäden anrichten, wie Unfruchtbarkeit beim Mann oder eine Fehlentwicklung von heranwachsenden Kindern. In vielen Aspekten ist der Stoff vermeidbar, wie zum Beispiel das aufbewahren von Essenund Trinken in Glas-,Keramik- oder Edelstahlbehältern, dadurch wirkt man auch gleichzeitig der \glqq Wegwerfgesellschaft\grqq{} entgegen und zeigt sich Vorbildlich im Thema Nachhaltigkeit. Komplett ersetzen lässt sich BPA zur jetziger Zeit nicht, Beispielsweise basieren 75 \% aller weltweit verwendeten Epoxidharze auf Bisphenol A \cite{Epoxidharz} und bis ein geeigneter Ersatzstoff gefunden wird, wird sich vermutlich in nächster Zeit daran nichts ändern.\\
\\
Wichtig zu erwähnen sind die Behördlichen Regulierungen, die nach und nach immer strenger mit Bisphenol A umgehen, es ist wahrscheinlich nur eine Frage der Zeit bis es in die Liste der zulassungspflichtigen Stoffe der REACH aufgenommen wird.\\
\\
Aus der Sicht eines Politikers ist diese Regulierung dem Interview entnommen ein bisher großer Erfolg, da es meist sehr lange dauert etwas wie zum Beispiel neue Regelungen durch zu setzen, vor allem wenn dabei wirtschaftliche Einbuße zu erwarten sind.\\
\subsection{Persönliche Stellungnahme}
Mein persönliche Meinung ist, dass die Gefährlichen Eigenschaften von Bisphenol-A nicht unterschätzt werden sollten, weshalb ich die immer strenger werdenden Regulierungen begrüße. Bei der Frage, ob Bisphenol-A verboten werden sollte, sollte man die Risiken vom Nutzen abwägen und einen Blick in Richtung Forschung wenden, da diese ausschlaggebend für einen Ersatzstoff ist. Meiner Meinung nach würde eine Aufnahme in die Liste der zulassungspflichtigen Stoffe reichen, bis ein optimaler Ersatzstoff bei dem nachwiesen wurde, dass keine Risiken bestehen entdeckt wird.\\
\\
Das Thema Bisphenol A ist zwar schon ein älteres, jedoch immer noch ein aktuelles Thema.
Meiner Meinung nach sollten die Normalverbraucher mehr aufgeklärt werden um auch zu verstehen, wieso man auf Produkte wie Polycarbonat Flaschen verzichten sollte und am besten eine Glas-Flasche als Trinkbehälter nutzen sollte. Ich selbst habe nach bzw. während der Ausarbeitung erst festgestellt, wie sehr Bisphenol A in Produkten, die ich in meinem alltäglichen Leben benutze integriert ist. Seit dem ich aufgeklärter über dieses Thema bin gebe ich auch mehr Acht, auf Bisphenol A enthaltene Produkte so gut es geht zu verzichten.
