\section{Fazit}
\subsection{Erkenntnisse der Ausarbeitung}
Bisphenol-A zeigt sich als alltäglicher Begleiter in einigen damit produzierten Kunststoffen. Durch bestimmte Bedingungen löst es sich aus den Produkten und wird frei. Dadurch kann es in den Menschlichen Körper gelangen und erhebliche Schäden anrichten, wie Unfruchtbarkeit beim Mann oder Fehlentwicklung von heranwachsenden Kindern. In vielen Aspekten ist der Stoff vermeidbar, jedoch genau so in einigen anderen noch nicht.
\subsection{Persönliche Stellungnahme}
Die Gefährlichen Eigenschaften von Bisphenol-A sollten nicht unterschätzt werden, weshalb ich die immer strenger werdenden Regulierungen begrüße. Bei der Frage, ob Bisphenol-A verboten werden sollte, sollte man die Risiken vom Nutzen abwägen und einen Blick in Richtung Forschung wenden, da diese ausschlaggebend für einen Ersatzstoff ist. Meiner Meinung nach würde eine Aufnahme in die Liste der zulassungspflichtigen Stoffe reichen, bis ein optimaler Ersatzstoff bei dem nachwiesen wurde das keine Risiken bestehen entdeckt wird.
