\section{Politisches}
\subsection{Bewertung der Wirtschaftlichen Vorteile und Risiken für die Umwelt}
Der einzige aber auch ausschlaggebende Vorteil, welcher der Stoff Bisphenol-A mitsich bringt,
sind die Produkte, in denen Bisphenol-A eine Rolle spielt. Vorallem wie schon im
Abschnitt 4.1 erwähnt, sind es die Produkte Polycarbonat und Epoxidharz, welche
mitlerweile nicht mehr aus dem Alltag weg zu denken sind. Da Ersatzstoffe meist teurer
in der Produktion sind und man noch nicht ganz klar ist ob Ersatzstoffe
keine Risiken darstellen, da zum Beispiel der Ersatzstoff Bisphenol-S laut einer
kritisch zu betrachtenden Forschung ein endokrines Potenzial besitzt\cite{Ersatzstoffe},
stellt das keine optimale Lösung dar.
Laut \cite{Umweltbundesamt2016} lässt sich schon seit den 70er Jahren Bisphenol-A
in Gewässern und Flüssen auffinden, sowie ein erhöhtes Vorkommen in den letzten Jahren.
Wie in behördlichen Messungen ausführlich beschrieben wird, ist das Vorkommen von
Bisphenol-A in der Umwelt unwiderlegbar. Bis zu einem Grenzwert an Bisphenol-A ist es
weitgehend nicht schädlich für Mensch und Umwelt, dass Problem hierbei ist jedoch
das dieser Grenzwert immer wieder aufgrund von mangel an Information neu bestimmt wird.
Der Mangel an Information zeigt sich auch in der Risiko-Bewertung von Bisphenol-A, da sich keine abschließende Beurteilung abgeben lässt wie hoch das Risiko genau für die Umwelt ist, laut einer Neubewertung der EFSA am 21.Januar 2015 sei bei derzeitigem Verbrauch kein Gesundheitsrisiko vorhanden, jedoch sagen die folgenden Regulierungen etwas anderes aus.
\subsection{Behördliche Regulierungen}
\begin{tabular} [h]{l|l|l}
Datum & Staat(en) & Regulierung/Entscheid\\
\hline
Februar 2009 & Schweiz & Bisphenol-A stellt bei der Einnahme durch Lebensmittel\\
& & kein Risiko dar. \\
\hline
März 2009 & Australien und Neuseeland & Bisphenol-A nicht mehr in Babyflaschen. \\
\hline
Anfang 2010 & Frankreich & Verbot in Babyflaschen.\\
\hline
1. März 2011 & Europäische Union & Verbot für Produktion und ab 1. Juni auch der Verkauf von\\ &&Babyflaschen die Bisphenol-A enthalten. \\
\hline
Ab 2013 & Frankreich & Verbot in Lebensmittelverpackungen für unter 3 Jahre alte Kinder \\
\hline
2014 & Frankreich & Einreichen eines Restriktionsdossier für Thermopapier \\
\hline
Ab 2015 & Frankreich & Verbot in Lebensmittelverpackungen jeglicher Art \\
\hline
2017 & Schweiz & Verbot von Polycarbonat-Babyflaschen\\
\hline
1. März 2018 & Europäische Union & Einstufung als reproduktionstoxisch: 1B \\
\hline
ab 2020 & Europäische Union & Verbot für die Verwendung von Bisphenol-A als Beschichtung von\\
&&Thermopapieren\\
\end{tabular}
Die Daten wurden von \cite{Wikipedia} entnommen.\\
Wie in der Tabelle gut zu sehen ist, erweist sich Frankreich, wenn es um die Regulierung geht als sehr Fleißig und handelt Vorbildlich. Überraschend ist zu sehen, dass die Schweiz hier hinterher hinkt und erst 2017 ein Verbot für Polycarbonat-Babyflaschen erteilt haben. Deutschland wurde speziell nicht erwähnt, da Deutschland sich immer den Vorgaben der Europäischen Union angepasst haben.
\subsection{Interview mit Dr. Hüttenhofer}
\begin{tabular} [h]{l|l}
Fragen & Antworten\\
\hline
Sie haben im Gebiet der Kunststoffe &Er sei schon immer politisch interessiert gewesen, jedoch\\
(Katalysatoren) studiert, wie sind Sie zum& ist er erst seit 3 Jahren Mitglied einer Partei (die Grünen)\\
Entschluss gekommen Politiker zu werden? & Partei ergriffen, hat er weil er sich immer stärker dazu\\
& gedrängt fühlte mit seinem Fachwissen weiter zu helfen.\\
\hline
Würden Sie mit dem jetzigen Wissen darüber, & Ja, jedoch bestimmte Forschung nicht im Auftrag von\\
was Kunststoffe anrichten können trotzdem &ethnischen "no go's". Polymeer werkstoffe werden wir immer\\
wieder in diesem Bereich studieren? &brauchen. Durch die Kreislaufwirtschaft kann man sie wieder\\
&abbauen, sprich man sollte sie recyclen. Das Problem ist\\
&nicht die Forschung, sondern der Umgang, weshalb man\\
&mehr chemische Prozesse für umweltfreundlichere Stoffe\\
& machen sollte.\\
\hline
Was sagen Sie zur Situation in & Der Ursprung von einem Waldbrand ist multikausal\\
Australien (Waldbrände)?&die Intensität wird jedoch klar durch den Klimawandel\\
&verstärkt.\\
\hline
BPA war Urprünglich als ein hormoneller Stoff&Er ist Überrascht, es zeigt eine Lücke in der Auswertung.\\
mit einer Östrogenen Wirkung vorgesehen, was &Gegenwärtig gibt es eine liste der "Reach" für gefährliche \\
halten Sie davon bzw. was hätte man besser&Stoffe.\\
machen können?&\\
\hline
Dieses Jahr wurde BPA in Thermopapieren für & Die Wirkung auf die Umwelt ist unbedenklich, weshalb\\
verboten erklärt, reicht dieses Verbot oder wie &man einen bestimmten Grenzwert haben sollte. Hierbei,\\
weit sollten Verbote ihrer Meinung nach gehen? &sollte wissenschaftlich ran gegangen werden,\\
&indem man sich die Anwendungsgebiete ansieht und \\
&die Risiken von dem Nutzen abwägt. Außerdem, sollte \\
&es strenge Verbote geben um so viel wie möglich zu \\
&verbieten. Zum Beispiel sollte es im Verpackungsbereich\\
&verboten werden.\\
\hline
Wie nachhaltig, sehen Sie den Umgang mit &Die Menge an hergestellten Produkten ist eindeutig zu \\
Kunststoffen in Deutschland? Und was sollte &extrem und im Verbrauch sind wir deutschen schlecht,\\
schnellstmöglich verändert werden? &da die Wiederverwendung meist nicht wirtschaftlich ist.\\
&Man sollte hergestelle Produkte solange wie möglich \\
&verwenden, recycling verbessern und monomere\\
& wiederherstellen $->$ Kreislaufwirtschaft, außerdem \\
&sollten abgaben auf Kunststoffmüll und die CO2 Steuer\\
& ausgedehnt werden.\\
\hline
Was belastet Sie in ihrer Arbeit als &Schon alleine das notwendigste zu erreichen stellt\\
(Ehrenamtlicher) Politiker am meisten?&eine Herausforderung, da Interessen oft emotional\\
&oder irrational sind.\\
&Fake news sind ein Problem.\\
&Es braucht sehr radikale Maßnahmen, jedoch bekommt\\
&dadurch weniger Wähler. Man soll das Ziel vor den\\
&Augen nicht verlieren. Die Arbeit als Politiker ist\\
&oft frustrierend.\\



\end{tabular}


