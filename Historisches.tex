\section{Was ist Bisphenol-A?}
\subsection{Entstehung und Anfangszweck}
Die britischen Biochemiker Edward Charles Dodds und Wilfrid Lawson suchten 
im Jahr 1936 nach Chemikalien, welche das nätürliche Sexualhormon Ötrogen in medizinischen 
Therapien ersetzen kann, da dieses noch aus dem Urin von trächtigen Stuten 
aufbereitet werden musste, was sehr kostenaufwendig war \cite[]{Umweltbundesamt2010}. 
Wie man Gegenwärtig jedoch sieht ist Bisphenol-A kein Pharmazie-Produkt, da die selben Forscher
später weitaus bessere synthetische Östrogene entdeckten, weshalb Bisphenol-A im Bereich
der Hormontherapie ausfiel \cite{Wikipedia}. 

\subsection{Hormonelle Wirkung im humanen Körper}
Stoffe, welche wenn sie ab einer bestimmten Konzentration in ein Hormonsystem gelangen,
dieses verändern und somit die entwickling der Embryonen stören bzw. die Fortpflanzung
beeinflussen, werden \textit{Endokrine Disruptoren} genannt.
Durch das Andocken, der Stoffe an die für eigentlich nätürlichen Sexualhormone 
vorgesehenen Rezeptoren, werden diese entweder aktiviert oder gehemmt \cite{Umweltbundesamt2010}.
Bisphenol-A zeigt Untersuchungen zufolge, dass weibliche Sexualhormone gestärkt werden und 
gleichzeitig männliche Sexualhormone gehemmt werden.
Im humanen Körper wird Bisphenol-A zwar sehr schnell zu Bisphenol-A-Glucuronid und 
Bisphenol-A-Sulfat metabolisiert $($abbgebaut$)$ und somit unschädlich gemacht, allerdings
können in menschlichen Zellgeweben wie Hoden und Mutterkuchen die unmetabolisierte wirksame
Form $(auch freie Form genannt)$ freigesetzt werden \cite{Umweltbundesamt2010}.

\subsection{Darstellung}
\subsection{Herstellung}
\subsection{Freisetzung}
