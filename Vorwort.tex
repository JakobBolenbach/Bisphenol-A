\section{Einleitung}
Was ist Bisphenol-A, wie wird es hergestellt und wie ist es möglich, dass es eine so große Karriere als Industrie-Chemikalie machen konnte? BPA sollte ursprünglich das Sexualhormon \glqq Östrogen\grqq{} ersetzen, wieso dies gegenwärtig nicht der Fall ist, war eine meiner Leitfragen bei der Ausarbeitung, denn BPA betrifft uns ja alle, da wir täglich auch meist unwissentlich diesem Stoff in scheinbar \glqq harmlosen\grqq{} Produkten ausgesetzt sind. Interessant zu wissen ist auch, dass dieser Stoff normalerweise sofort von den dafür spezialisierten Behörden hätte beschränkt werden sollen, es jedoch vermutlich einer Lücke bei der Auswertung zum Vorteil der Vermarktung von BPA unterlag. BPA hat eine Hormonelle Wirkung, aber gefährlich für den menschlichen Körper ist die Freisetzung aus einem Produkt und nicht das Beinhalten des Stoffes, denn bei keiner Freisetzung gäbe es auch keine ernst zu nehmende Gefahr, leider ist dies offensichtlich nicht der Fall, weshalb in der Ausarbeitung darauf speziell eingegangen wird. Um die Freisetzung aus chemischer Sicht besser verstehen zu können, werden davor die Darstellung und die Zwischenschritte zum Endprodukt näher betrachtet. Außerdem wird noch auf das Vorkommen und die Vermeidbarkeit eingegangen, um dem Leser auf mögliche Alternativen hinzuweisen. Der praktische Teil dieser Ausarbeitung und \glqq Highlight\grqq{} ist das Interview mit Dr. Hüttenhofer ein ehrenamtlicher Politiker und Klimaschutz-Aktivist für Bündnis 90/die Grünen.
\\Die Seminararbeit ist folgendermaßen aufgebaut: Zuerst wird sich mit der Erklärung rund um BPA befasst, was es ist und wie es sich auf unseren Alltag ausgewirkt hat. Dabei wird geschaut wie es heutzutage eingesetzt wird und welche Vor-und Nachteile es aus wirtschaftlicher Sicht hat. Zusätzlich zum oben genannten Interview werden politische Themen bearbeitet, in denen der Aspekt aus der Sicht der Behörden vorgestellt wird, sowie die Risiken für die Umwelt. Abschließend folgt ein Fazit, welches die Ausarbeitung mit den wichtigsten Aspekten Zusammenfasst und eine persönliche Stellungnahme beinhaltet.
\\Fakten und Daten, werden mit Hilfe der Literatur bewiesen.
